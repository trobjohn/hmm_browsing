\documentclass[11pt]{article}

% ----------- PACKAGES -----------
\usepackage[utf8]{inputenc}
\usepackage[T1]{fontenc}
\usepackage{microtype}
\usepackage{lmodern} % clean font
\usepackage{amsmath, amssymb, amsthm}
\usepackage{graphicx}
\usepackage{booktabs}
\usepackage{hyperref}
\usepackage{natbib}
\usepackage{geometry}
% ----------- PAGE SETUP -----------
\geometry{margin=1in}
\setlength{\parskip}{0.6em}
\setlength{\parindent}{0pt}
\linespread{1.15}

% ----------- HYPERREF COLORS -----------
\hypersetup{
  colorlinks=true,
  linkcolor=blue,
  citecolor=magenta,
  urlcolor=blue
}

% ----------- THEOREMS -----------
\newtheorem{theorem}{Theorem}
\newtheorem{lemma}{Lemma}
\newtheorem{definition}{Definition}
\newtheorem{proposition}{Proposition}
\newtheorem{corollary}{Corollary}

% ----------- TITLE INFO -----------
\title{Title of the Paper}
\author{Terry Johnson\\
\small University of Virginia \\
\small \texttt{terry.johnson@example.edu}}
\date{\today}

% ----------- BEGIN DOCUMENT -----------
\begin{document}

\section*{Model}

\begin{itemize}
\item{ The \emph{ideology space} is $\mathcal{I} = \{ -1, ..., 0, ..., +1\}$ with $L$ elements }
\item{ $\theta_{it} \in \mathcal{I}$ is the ideology of user $i$ at time $t$}
\item{A page is a tuple, $(v_j, \mu_{j}, \sigma_j^2, \alpha_j,\beta_j)$ where
\begin{itemize}
\item{The variables $(\mu_j, \sigma_j^2)$ govern the production of its content when visited. This distribution is a quantized normal, $N_j = \Phi((x-\mu_j)/\sigma_j)$, over $\mathcal{I}$.}
\item{The variables $(\alpha_j, \beta_j)$ govern the influence it has over users after observing content. This distribution is a quantized Kamuraswamy, $K_j(x) = 1 - (1-x^\alpha_j)^{\beta_j}$, over $\mathcal{I}$}
\end{itemize}}
\item{The user's expected utility from visiting page $j$ is 
$$
U_j(\theta_{it}) = \int_{\mathbb{R}} v_j - \left|\dfrac{\theta_{it}-n}{w} \right|^\tau d N_j(n)
$$
where $(u_0, w, \tau)$ are non-negative parameters. A user visits page $j$ with probability
$$
p[j|\theta_{it}] = \dfrac{e^{U_j(\theta_{it})}}{1 + \sum_{j'} e^{U_{j'}(\theta_{it})}} \mathbb{I}\{ U_j(\theta_{it}) \geq 0  \},
$$ 
and ends the browsing session with probability
$$
p[EOS|\theta_{it}] = \dfrac{1}{1 + \sum_{j'} e^{U_{j'}(\theta_{it})}}
$$ }
\item{After a user with ideology $\theta_{it}$ visits page $j$ and observes content $n_{jt}$, their ideology transitions to a point in the interval $[\theta_{it}\wedge n_{jt}, \theta_{it}\vee n_{jt}] \cap \mathcal{I}$. This transition occurs as Kamuraswamy, with
$$
x_{ij} = \dfrac{}{}
$$
}
\end{itemize}
So a model is a tuple $(\tau, w)$ of consumer parameters, and tuples $\{(v_j, \mu_j, \sigma^2_j, \alpha_j, \beta_j)\}_{j=1,...,J}$, all to be estimated from transition data at the user level.


\section*{Computation}

A path is a sequence $\pi_i



\end{document}
